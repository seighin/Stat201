\documentclass{article}

\usepackage{shyne}

% document format
\topmargin 0in
\oddsidemargin 0in
\evensidemargin 0in
\headheight 0in
\headsep 0in
\topskip 0in
\textheight 9in
\textwidth 6.5in
\linespread{1.3}

\begin{document}


\begin{flushleft}
\section*{Group Work - Chapter 3}
\paragraph{1} The file ``max\_temp\_dec17.csv" on D2L contains the daily high temperatures (in F) for December. 
\begin{itemize}
\item [(a)] Find the mean, median, mode and midrange of the sample. What is the most appropriate measure of center for this data?
\vspace{2.25in}
\item[(b)] Find the range, variance and standard deviation of the sample. Be sure to include the correct units for each measure.

\vspace{2.25in}
\item[(c)] Within the data set, what percentile is 46 degrees? What temperature is the 10th percentile?
\end{itemize}



\newpage
\paragraph{2} The file ``mpls\_home\_sales.csv" on D2L contains the adjusted sale prices (in dollars) of a sample of home sold in Minneapolis in 2016. 
\begin{itemize}
\item [(a)] Find the mean, median, mode and midrange of the sample. What is the most appropriate measure of center for this data?
\vspace{2.25in}
\item[(b)] Find the range, variance and standard deviation of the sample. Be sure to include the correct units for each measure.
\vspace{2.25in}
\item[(c)] Find the 5 number summary of the sample and create a boxplot. What does the boxplot tell you about the distribution of the data.
\end{itemize}

\newpage
\paragraph{3} The file ``heights.csv" on D2L contains simulated heights (in inches) of 25 men and 25 women in a statistics class. 
\begin{itemize}
\item [(a)] Find the mean and median of both samples. What is the most appropriate measure of center for this data?
\vspace{2.25in}
\item[(b)] Find standard deviation of both samples.
\vspace{2.25in}
\item[(c)] Suppose two new students join the class. One is a woman who is 71 inches tall and one is a man who is 74 inches tall. Calculate z-scores for both. Who is taller for their gender? Are either of them unusually tall?
\end{itemize}


\end{flushleft}
\end{document}