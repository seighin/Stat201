\documentclass{article}

\usepackage{shyne}

% document format
\topmargin 0in
\oddsidemargin 0in
\evensidemargin 0in
\headheight 0in
\headsep 0in
\topskip 0in
\textheight 9in
\textwidth 6.5in
\linespread{1.3}

\begin{document}

\begin{flushleft}
\section*{Group Work - Chapter 4}
\paragraph{1} Consider rolling a fair six-sided die.
\begin{enumalpha}
\item Let event $A$ be rolling an even number. What is a trial for this scenario? What is the sample space? Is $A$ a simple event? What is $P(A)$? What is $\bar A$, the complement of $A$? What is $P(\bar A)$? Is $A$ unlikely? Is $A$ unusual?
\vspace{2.5in}
\item Let event $A$ be rolling an even number. Let event $B$ be rolling a 3. Are events $A$ and $B$ disjoint? What is $P(A \text{ or } B)$?
\vspace{2.5in}
\item Consider rolling a die twice. Let event $A$ be getting an even number on the first roll. Let event $B$ be getting 5 or more on the second roll. Are events $A$ and $B$ independent? What is $P(A \text { and } B)$?
\end{enumalpha}

\newpage
\paragraph{2} Consider a standard deck of playing cards... 52 cards, 4 suits of 13 cards each, 3 cards of each suit are face cards, 2 suits are black (clubs and spades) and 2 are red (hearts and diamonds).
\begin{enumalpha}
\item Let event $A$ be drawing a random card that is a diamond. What is a trial for this scenario? What is the sample space? Is $A$ a simple event? What is $P(A)$? What is $\bar A$, the complement of $A$? What is $P(\bar A)$? Is $A$ unlikely? Is $A$ unusual?
\vspace{2.5in}
\item Let event $A$ be drawing a random card that is a diamond. Let event $B$ be drawing a random card that is a face card. Are events $A$ and $B$ disjoint? What is $P(A \text{ or } B)$?
\vspace{2.5in}
\item Consider drawing three cards. Let event $A$ be the first card is a heart. Let event $B$ be the second card is black. Let event $C$ be the third card is a club. Are events $A$, $B$ and $C$ independent? What is $P(A \text { and } B \text { and } C)$?

\end{enumalpha}

\newpage
\paragraph{3} The data set ``hair\_eye.csv" on D2L contains the hair and eye colors, as well as sex, of a sample of statistics students. Below is a table showing the distributions of students by eye color and gender.\\
\medskip
\renewcommand{\arraystretch}{1}
{\centering
\begin{tabular}{ r| c c c c}
& \multicolumn{4}{c}{Eye color}\\
Gender & Blue & Brown & Green & Hazel\\
\hline
 Female &  114   &  122  &  31  &  46\\
 Male   & 101  &  98  &  33  &  47
\end{tabular}
\par}
\begin{enumalpha}
\item Let event $A$ be a randomly selected student having green eyes. What is a trial for this scenario? What is the sample space? Is $A$ a simple event? What is $P(A)$? What is $\bar A$, the complement of $A$? What is $P(\bar A)$? Is $A$ unlikely? Is $A$ unusual?
\vspace{2.25in}
\item Let event $A$ be a randomly selecting a student with brown or blue eyes. Let event $B$ be a randomly selecting a female student. Are events $A$ and $B$ disjoint? What is $P(A \text{ or } B)$?
\vspace{2.25in}
\item Consider randomly selecting two students. Let event $A$ be the first student has blue eyes. Let event $B$ be the second student has hazel eyes. Are events $A$ and $B$ independent? What is $P(A \text { and } B)$?

\end{enumalpha}

\newpage
\paragraph{4} The table below is the distribution of students from the ``hair\_eye.csv" data set by eye color and hair color.\\
\medskip
\renewcommand{\arraystretch}{1}
{\centering
\begin{tabular}{ r| c c c c}
& \multicolumn{4}{c}{Eye color}\\
Hair color & Blue & Brown & Green & Hazel\\
\hline
  Black &  20  &  68  &   5  &  15\\
 Blond  & 94   &  7  &  16  &  10\\
 Brown &  84  & 119  &  29  &  54\\
 Red   &  17  &  26  &  14  &  14
\end{tabular}
\par}

\begin{enumalpha}
\item Suppose 5 students are randomly selected. What is the probability that at least one of them has red hair?
\vspace{2.25in}
\item Suppose one student is selected randomly. Assuming the student has black hair, what is the probability the student has brown eyes? Is having brown eyes independent of having black hair?
\vspace{2.25in}
\item Repeat part (b), but assume the student has brown hair.

\end{enumalpha}


\end{flushleft}
\end{document}