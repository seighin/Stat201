\documentclass{article}

\usepackage{shyne}

% document format
\topmargin 0in
\oddsidemargin 0in
\evensidemargin 0in
\headheight 0in
\headsep 0in
\topskip 0in
\textheight 9in
\textwidth 6.5in
\linespread{1.3}

\begin{document}

\begin{flushleft}
\section*{Group Work - Chapter 2}
\paragraph{1} The high temperatures in Saint Paul for each day in December are below. The data can also be found on D2L as ``max\_temps\_dec17.csv". Use classes -9 to 0, 1 to 10, 11 to 20 etc. 
\[51 ,\, 46 ,\, 49 ,\, 57 ,\, 27 ,\, 19 ,\, 22 ,\, 31 ,\, 27 ,\, 29 ,\, 35 ,\, 24 ,\, 29 ,\, 24 ,\, 29 ,\, 30 ,\, 28 ,\, 42 ,\, 38 ,\, 23 ,\, 24 ,\, 25 ,\, 22 ,\, 22  ,\, 3  ,\, 0  ,\, 4 ,\, 11  ,\, 7 ,\, -6 ,\, -8\]
\begin{itemize}
\item [(a)] Build a frequency table of the data. Does the data appear to have a normal distribution? Are there outliers? Add relative frequency and cumulative frequency tables.
\vspace{3in}
\item[(b)] Sketch a histogram of the data. Does the histogram show a normal distribution? If not, which characteristic of normal distributions does it violate?
\end{itemize}



\newpage
\paragraph{2} The maximum monthly fine particulate matter pollution in Minnesota from 2007 to 2009 is below. The data can be found on D2L as ``max\_air\_pol.csv". Use classes 10-15, 15-20, 20-25, etc. 

\[  25.82 ,\,  28.87 ,\, 36.18 ,\, 21.22 ,\, 25.94 ,\, 25.10 ,\, 24.49 ,\, 36.94 ,\, 23.20 ,\, 25.72 ,\, 27.94 ,\, 43.34 \] 
\[31.46 ,\, 38.32 ,\, 24.82 ,\, 22.78 ,\, 28.48 ,\, 24.27 ,\, 24.26 ,\, 15.52 ,\, 22.88 ,\, 15.95 ,\, 14.48 ,\, 23.83 \]
\[ 28.72 ,\, 25.87 ,\, 22.38 ,\, 21.21 ,\, 19.90 ,\, 32.25 ,\, 24.69 ,\, 23.14 ,\, 23.81 ,\, 17.05 ,\, 24.09 ,\, 49.88 \]
\begin{itemize}
\item [(a)] Build a frequency table of the data. Does the data appear to have a normal distribution? Add relative frequency and cumulative frequency tables.
\vspace{3in}
\item[(b)] Sketch a histogram of the data. Does the histogram show a normal distribution? If not, which characteristic of normal distributions does it violate?
\end{itemize}


\newpage
\paragraph{3} For each scenario below, identify some of the graphs that would be appropriate for this data. Create one appropriate graph and use it to answer the research question.

\begin{itemize}
\item [(a)] The data set ``faithful.csv" contains a sample of eruptions of the Old Faithful geyser in Yellowstone National Park. Length of eruption (in minutes) and waiting time until the next eruption (in minutes) are recorded. Researchers want to know if the is a relationship between eruption lengths and waiting times.
\vspace{2in}
\item[(b)] The data set ``hair\_eye.csv" contains the hair and eye color of 592 statistics students. The math department would like to know what is the most common eye color. What is the least common eye color?
\vspace{2in}
\item[(c)] The data set ``max\_air\_pol.csv" contains the monthly maximum fine matter particulate air pollution from 2007 to 2009. Researchers would like to know if there is a pattern in air pollution over time.
\end{itemize}


\end{flushleft}
\end{document}