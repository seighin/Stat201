\documentclass{article}

\usepackage{shyne}

% document format
\topmargin 0in
\oddsidemargin 0in
\evensidemargin 0in
\headheight 0in
\headsep 0in
\topskip 0in
\textheight 9in
\textwidth 6.5in
\linespread{1.3}

\begin{document}

\begin{flushleft}
\section*{Group Work - Week 6}
\paragraph{1}
\begin{enumalpha}
\item The time it takes to finish a statistics midterm in a certain class is uniformly distributed between 1 and 2 hours (60 to 120 minutes). What is the probability that a randomly selected student will finish the exam in less than 75 minutes? What is the probability they will complete it between 100 and 110 minutes?
\vspace{3.25in}
\item The scores of the midterm are normally distributed and reported as z-scores. What is the probability of a random student getting a grade greater than $z = -0.5$? What is the grade ($z$-score) that separates the bottom 90\% of the class from the top 10\%?
\end{enumalpha}



\newpage
\paragraph{2} A new breakfast cereal ``Super Fruity Taco Bombs" is packaged by a machine which puts the cereal in the bag. The amount loaded by the machine is normally distributed with a mean of 14.5 ounces with a standard deviation of 1.15 ounces. A bag is rejected if it weighs less than 13 ounces.
\begin{enumalpha}
\item What is the proportion of bags of cereal that are rejected? What is the probability that a bag weighs between 15 and 16 ounces?
\vspace{3.25in}
\item The company doesn't want to check each bag individually, so it begins weighing cases of 16 bags of cereal, rejecting a case if the mean weight of the bags is less than 13 ounces. Can we apply the CLT to this procedure? What proportion of cases get rejected? Is this procedure fair to the consumer?
\end{enumalpha}

\newpage
\paragraph{3} A certain midwestern city has a population with ages that are normally distributed with a mean of 34.5 years and a standard deviation of 13.9 years.
\begin{enumalpha}
\item What is the probability of a randomly selected resident being younger that 25 years old? What proportion of the population is over 60?
\vspace{3.25in}
\item A marketing research firm wants to study characteristics of residents of the city. So they randomly select 9 residents to complete a survey. Is the CLT theorem applicable here? What is the probability the sample is ``young" (mean age 25 or under)? What is the probability to mean age is over 60?
\end{enumalpha}


\end{flushleft}
\end{document}