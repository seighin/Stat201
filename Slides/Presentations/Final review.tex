\documentclass[xcolor=table]{beamer}

\usepackage{shyne}

% Theme settings
\setbeamertemplate{navigation symbols}{}

\usetheme{Madrid}
\usefonttheme{structurebold}
\usefonttheme[onlymath]{serif}

\AtBeginSection[]
{ 	\begin{frame}{}

	{
	\usebeamerfont{frametitle}
	\begin{beamercolorbox}
		[wd={\textwidth}, center, sep=.2in, rounded=true, shadow=true]
		{frametitle}
	Chapter \thesection\\  \secname 
	\end{beamercolorbox}
	}
	
	\end{frame} 
}

\AtBeginSubsection[]
{ 	\begin{frame}{}

	{
	\usebeamerfont{frametitle}
	\begin{beamercolorbox}
		[wd={\textwidth}, center, sep=.2in, rounded=true, shadow=true]
		{frametitle}
	Section \thesection .\thesubsection\\  \subsecname 
	\end{beamercolorbox}
	}
	
	\end{frame} 
}

\title[Midterm Review]{Stat 201: Statistics I\\Midterm Review}
\author[M. Shyne]{}
\institute[Metro State]{\includegraphics[width=1.75in]{../images/metro_logo}}
\date[8/7/2018]{
\\ \bigskip \bigskip \includegraphics[width=.4in]{../images/cc_big}}


\begin{document}
\frame{\titlepage}

\begin{frame}{About the final exam}
\begin{block}{}
\large
\begin{itemize}
\item Available on MyStatLab following class on 12/3
\item Due on 12/10 by midnight
\item 23 questions, 100 points
\begin{itemize}
\item 5 questions, 18 points from chapters 3 - 6
\item 18 questions, 82 points from chapters 7 - 11
\end{itemize}
\item Every question on exam has been a homework question, though the details will likely be different
\item Time limit: 4 hours, must be completed in one sitting
\item Can use any resource (book, notes, internet), except other people
\end{itemize}
\end{block}
\end{frame}


\begin{frame}{Chapter 3}
\begin{block}{}
\large
\begin{itemize}
\item From a set of data, find:
\begin{itemize}
\item Mean
\item Median
\item Mode
\item Midrange
\item Range
\item Variance
\item Standard deviation
\end{itemize}
\end{itemize}
\end{block}
\end{frame}

\begin{frame}{Chapter 4}
\begin{block}{}
\large
\begin{itemize}
\item Calculate probabilities:
\begin{itemize}
\item From proportions (3 in 12)
\item From a contingency table
\item Complements
\item \bt{Addition rule}
\item \bt{Multiplication rule}
\item Complex events ("At least one...")
\item Conditional events
\end{itemize}
\end{itemize}
\end{block}
\end{frame}

\begin{frame}{Chapter 5}
\begin{block}{}
\large
\begin{itemize}
\item Find probability of event from a binomial distribution
\end{itemize}
\end{block}
\end{frame}

\begin{frame}{Chapter 6}
\begin{block}{}
\large
\begin{itemize}
\item Find probability from standard normal, $z$, distribution
\item Find $z$-score which corresponds to given probability
\item Find probability of event from a non-standard normal distribution
\item Find value from non-standard normal distribution which corresponds to given probability
\end{itemize}
\end{block}
\end{frame}

\begin{frame}{Chapter 7}
\begin{block}{}
\large
\begin{itemize}
\item Estimate a population proportion:
\begin{itemize}
\item Find a confidence interval 
\item Correctly interpret a confidence interval
\item Find sample size for desired margin of error
\begin{itemize}
\item Known and unknown estimated proportion
\end{itemize}
\end{itemize}
\item Estimate a population mean:
\begin{itemize}
\item Find a confidence interval 
\item Correctly interpret a confidence interval
\item Find sample size for desired margin of error
\end{itemize}
\end{itemize}
\end{block}
\end{frame}

\begin{frame}{Chapter 8}
\begin{block}{}
\large
\begin{itemize}
\item Hypothesis testing:
\begin{itemize}
\item Identify the null and alternative hypotheses
\item Calculate a test statistic and p-value
\item Make a decision based on p-value and significance level
\item State conclusion in terms of research question
\end{itemize}
\item Understand type I and type II errors
\item Test a claim about population proportion
\item Test a claim about population mean
\end{itemize}
\end{block}
\end{frame}

\begin{frame}{Chapter 9}
\begin{block}{}
\large
\begin{itemize}
\item Test a claim about two population proportions
\item Test a claim about two population means using two independent samples
\item Test a claim about the difference between populations using samples of paired data
\item For all tests, construct the appropriate confidence interval to test claims
\end{itemize}
\end{block}
\end{frame}

\begin{frame}{Chapter 10}
\begin{block}{}
\large
\begin{itemize}
\item Correlation:
\begin{itemize}
\item Identify linear correlation vs. non-linear correlation vs. no correlation
\item Estimate linear correlation from scatterplot
\item Calculate correlation coefficient of a sample
\item Test whether population correlation parameter $\rho$ is zero or not
\end{itemize}
\item Regression:
\begin{itemize}
\item Calculate regression equation from sample
\item Make predictions for the response variable given a predictor value and regression results
\end{itemize}
\end{itemize}
\end{block}
\end{frame}

\begin{frame}{Chapter 11}
\begin{block}{}
\large
\begin{itemize}
\item Test the fit of a sample frequency distribution to an expected distribution
\item Test independence of two factors using a sample contingency table

\end{itemize}
\end{block}
\end{frame}




\end{document}