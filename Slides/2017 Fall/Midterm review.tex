\documentclass[xcolor=table]{beamer}

\usepackage{shyne}

% Theme settings
\setbeamertemplate{navigation symbols}{}

\usetheme{Madrid}
\usefonttheme{structurebold}
\usefonttheme[onlymath]{serif}

\AtBeginSection[]
{ 	\begin{frame}{}

	{
	\usebeamerfont{frametitle}
	\begin{beamercolorbox}
		[wd={\textwidth}, center, sep=.2in, rounded=true, shadow=true]
		{frametitle}
	Chapter \thesection\\  \secname 
	\end{beamercolorbox}
	}
	
	\end{frame} 
}

\AtBeginSubsection[]
{ 	\begin{frame}{}

	{
	\usebeamerfont{frametitle}
	\begin{beamercolorbox}
		[wd={\textwidth}, center, sep=.2in, rounded=true, shadow=true]
		{frametitle}
	Section \thesection .\thesubsection\\  \subsecname 
	\end{beamercolorbox}
	}
	
	\end{frame} 
}

\title[Midterm Review]{Stat 201: Statistics I\\Midterm Review}
\author[M. Shyne]{}
\institute[Metro State]{\includegraphics[width=1.75in]{../images/metro_logo}}
\date[Oct 16, 2017]{October 16, 2017}


\begin{document}
\frame{\titlepage}

\begin{frame}{About the midterm exam}
\begin{block}{}
\begin{itemize}
\item Available on MyStatLab following class on 10/9
\item Due by end of day on 10/16
\item 23 questions, 100 points, covering chapters 1 - 6
\item Every question on exam has been a homework question, though the details will likely be different
\item Time limit: 4 hours, must be completed in one sitting
\item Can use any resource (book, notes, internet), except other people
\end{itemize}
\end{block}
\end{frame}

\begin{frame}{Chapter 1}
\begin{block}{}
\large
\begin{itemize}
\item Know the difference between a parameter and a statistic
\item Identify type of variable
\begin{itemize}
\item Quantitative: Discrete or continuous
\item Categorical (Qualitative)
\end{itemize}
\item Identify levels of measurement
\begin{itemize}
\item Nominal
\item Ordinal
\item Interval
\item Ratio
\end{itemize}
\item Identify type of study
\begin{itemize}
\item Experimental
\item Observational
\end{itemize}
\end{itemize}
\end{block}

\end{frame}

\begin{frame}{Chapter 2}
\begin{block}{}
\large
\begin{itemize}
\item Identify a normal distribution from a frequency table
\item Build a frequency table from data
\item Build a cumulative frequency table from a frequency table
\item Identify a histogram from a frequency table and assess normality
\item Types of graphs, identify proper graph
\begin{itemize}
\item Scatterplot
\item Time series
\item Dotplot
\item Stem-and-leaf plot
\item Bar chart
\item Pie chart
\item Pareto chart
\end{itemize}
\end{itemize}
\end{block}
\end{frame}

\begin{frame}{Chapter 3}
\begin{block}{}
\large
\begin{itemize}
\item From a set of data, find (with proper units):
\begin{itemize}
\item Mean
\item Median
\item Mode
\item Midrange
\item Range
\item Variance
\item Standard deviation
\end{itemize}
\item Calculate $z$-scores from mean and standard deviation
\item Identify unusual values as more than two SDs from mean
\item Calculate 5 number summary and identify corresponding boxplot
\end{itemize}
\end{block}
\end{frame}

\begin{frame}{Chapter 4}
\begin{block}{}
\large
\begin{itemize}
\item Calculate probabilities:
\begin{itemize}
\item From proportions (3 in 12)
\item From a 2 $\times$ 2 contingency table
\item Complements
\item Addition rule
\item Multiplication rule
\item Complex events ("At least one...")
\item Conditional events
\end{itemize}
\item Understand false positive and false negative
\item Identify disjoint events
\item Identify independent and dependent events
\end{itemize}
\end{block}
\end{frame}

\begin{frame}{Chapter 5}
\begin{block}{}
\large
\begin{itemize}
\item Find the mean and standard deviation for an arbitrary probability distribution
\item Find probability of event from a binomial distribution
\item Find the mean, standard deviation and boundaries for unusual values
\item Determine of a given value is unusual
\end{itemize}
\end{block}
\end{frame}

\begin{frame}{Chapter 6}
\begin{block}{}
\large
\begin{itemize}
\item Find probability of event from uniform distribution
\item Find probability from standard normal, $z$, distribution
\item Find $z$-score which corresponds to given probability
\item Find probability of event from a non-standard normal distribution
\item Find value from non-standard normal distribution which corresponds to given probability
\item Find probability of event from sampling distribution, using the Central Limit Theorem
\end{itemize}
\end{block}
\end{frame}

\end{document}