\documentclass{article}

\usepackage{shyne}

% document format
\topmargin 0in
\oddsidemargin 0in
\evensidemargin 0in
\headheight 0in
\headsep 0in
\topskip 0in
\textheight 9in
\textwidth 6.5in
\linespread{1.3}

\begin{document}

\begin{flushleft}
\section*{Group Work - Chapter 7}
\paragraph{1} Suppose it is known that 45\% of the general population believes that correlation implies causation.
\begin{enumalpha}
\item A survey taken of 60 students after completing a statistics course finds that 32\% of them believe that correlation implies causation. Find a 95\% confidence interval of the population proportion of students who have completed a statistics course who believe that correlation implies causation. Write it in both interval notation, $(L, U)$ or $L < p < U$, and in $\hat p \pm ME$ notation.
\vspace{2in}
\item Is the proportion of students who have completed a statistics course who believe that correlation implies causation different than the general population at a $\alpha = 0.05$ significance level? Did the statistics courses cause the difference, if it exists?

\vspace{2.25in}
\item If we wanted to known the population proportion of students who have completed a statistics course who believe that correlation implies causation within plus/minus 3\% with 90\% confidence, what sample size would be needed for a survey? Calculate with both unknown sample proportion and with proportion found in the earlier survey in part (a).
\end{enumalpha}



\newpage
\paragraph{2} Suppose winter daily maximum temperatures in Minnesota are known to be normally distributed with a standard deviation 12.5 \textdegree F. 
\begin{enumalpha}
\item A random sample of 16 winter day maximum temperatures has a sample mean of  14.3 \textdegree F. What is 90\% confidence interval for the population mean maximum temperature? Write it in both interval notation, $(L, U)$ or $L < \mu < U$, and in $\bar x \pm ME$ notation. 
\vspace{2.5in}
\item Suppose the file ``max\_temps\_dec17.csv" represents a random sample of max temperatures in December. What is a 99\% confidence interval based on this sample? (Don't have to write both forms.) Is the mean maximum December temperature different than the mean max winter temperature 14.3 \textdegree F as found in part (a) at a significance level of $\alpha = 0.01$?
\vspace{2.5in}
\item If we wanted to find the mean winter maximum temperature within plus/minus 1.5 \textdegree F with 95\% confidence, how many days would need to be sampled?
\end{enumalpha}



\end{flushleft}
\end{document}